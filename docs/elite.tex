\documentclass[
    narrow,
    fontstyle=light,
    babelparam=ngerman
]{elite}

% uncomment this in order to debug visual boxes (requires LuaLaTeX)
%\usepackage{lua-visual-debug}

\usepackage[utf8]{inputenc} % Sonderzeichen ermöglichen
\usepackage{blindtext}

% \usepackage[a4paper, margin=1in]{geometry}

\title{S32 \LaTeX{} Document-Class \enquote{elite}}
\author{Christian Schliz \& Julian Keck}
\date{\today}

\begin{document}

\maketitle

\vspace{2cm}

\defhint{Notes on I18n}{This document was originally intended to be used for documents
entirely in the german language. If you want to use this class for other languages, you
might need to redefine some commands and macros.}

\vspace{2cm}

\tableofcontents
\newpage

\section{General}

\begin{itemize}
    \item \code{\textbackslash{}tableofcontents}: print table of contents
\end{itemize}

\paragraph*{Typically end of document:}

\begin{itemize}
    \item \code{\textbackslash{}printindex}: print definition index
    \item \code{\textbackslash{}lstlistoflistings}: print code listing index
    \item \code{\textbackslash{}listoffigures}: print figure index
\end{itemize}

\paragraph*{Text Formatting}

\begin{itemize}
    \item \code{\textbackslash{}important}: \important{Important mark}
    \item \code{\textbackslash{}highlight}: \highlight{Text highlighting}
    \item \code{\textbackslash{}formalWord}: \formalWord{formal string definition}
    \item \code{\textbackslash{}code}: \code{same effect as \textbackslash{}texttt}
\end{itemize}

\section{Overview of all Font Sizes}

\begin{tabular}{rl}
    \code{tiny} & {\tiny tiny} \\
    \code{scriptsize} & {\scriptsize scriptsize} \\
    \code{footnotesize} & {\footnotesize footnotesize} \\
    \code{small} & {\small small} \\
    \code{normalsize} & {\normalsize normalsize} \\
    \code{large} & {\large large} \\
    \code{Large} & {\Large Large} \\
    \code{LARGE} & {\LARGE LARGE} \\
    \code{huge} & {\huge huge} \\
    \code{Huge} & {\Huge Huge}
\end{tabular}

\section*{Section}
\subsection*{Sub-Section}
\subsubsection*{Sub-Sub-Section}
\paragraph*{Paragraph}

\section{Elite Definitions}

\definition{Definition}{%
    A definition is created with the command \texttt{\textbackslash{}definition\{\textit{name}\}\{\textit{content}\}}.
    All styles are based on \texttt{\textbackslash{}abstractDefinition}
}

\defwarn{Warnung}{%
    is created by \texttt{\textbackslash{}defwarn\{\textit{Warnung}\}\{\textit{content}\}}
}

\defproblem{Problem}{%
    is created by \texttt{\textbackslash{}defproblem\{\textit{Problem}\}\{\textit{content}\}}
}

\defhint{Hinweis}{%
    is created by \texttt{\textbackslash{}defhint\{\textit{Hinweis}\}\{\textit{content}\}}
}

\newpage

\section{Include Graphics and Figures}

There are multiple ways to include images and figures inside of your doument.
When drawing attention towards an image, you should use the \code{\textbackslash{}fig} command:

\subsection{Regular Figures}

\fig[width=0.5\linewidth]{Prof. Dr. Thomas Worsch}{./worsch}

\begin{CodeListing}
\fig[width=0.5\linewidth]{Prof. Dr. Thomas Worsch}{./worsch}
\end{CodeListing}

\subsection{Wrapfigures}

\wrapfig[11]{r}{0.25\textwidth}{Thomas Worsch}{./worsch}

When including graphics which are not essential or just an added bonus on top of your content,
always use the \code{\textbackslash{}wrapfig} command. Wrapfigures place the image inside of
a box with the other content wrapping around said box. The \code{wrapfig} command has
five arguments:
\begin{enumerate}
    \item \textit{(optional)} Number of lines the box will wrap
    \item Width of the wrap box, e.g. \code{0.25\textbackslash{}textwidth})
    \item Alignment of the wrapper box \code{l}, \code{r} or \code{c}
    \item Width of the wrapper box, e.g. \code{0.25\textbackslash{}textwidth}
    \item Source of the included image
\end{enumerate}

\begin{CodeListing}
\wrapfig[10]{r}{0.25\textwidth}{Thomas Worsch}{./worsch}
\end{CodeListing}


\section{Algorithms and Code Segments}

When writing algorithms or other code-listings, you might want to move the whole block to the
next page if there are page breaks inbetween your lines of code.

\vspace{1em}
\definition{Fill breaking space}{Use \code{\textbackslash{}filbreak} before your code listing}

\begin{CodeListing}[caption=Code-Listing Positioning]
Imagine you are writing some code

and all of a sudden
...
...
your page breaks o_o

use \filbreak to avoid this problem ;D
\end{CodeListing}

In order to correctly format your code, use this syntax:
\begin{verbatim}
\begincode % use only if you want your code to float
           % (this might cause overlapping in some places)
\begin{CodeListing}[language=..., caption=...]

    ...

\end{CodeListing}
\endcode % if used, this must always go together with \begincode
\end{verbatim}

If you want the listing to be indexed, use \code{caption}, otherwise use \code{title}

\section*{Code-Listing lanuage highlight examples}

\subsection{language=pseudocode}

\begin{CodeListing}[language=pseudocode]
Function test(a: $\mathbb{N}$) {
    while n > 0 do
        something
        code...code    # yes
        code...code    // test
    done
}
\end{CodeListing}

\subsection{language=java}

\begincode
\begin{CodeListing}[language=java, caption=Beispiel-Code]
public class Example {
    public static void main(String args[]) {
        System.out.println("Hello, World!");
    }
}
\end{CodeListing}
\endcode

\subsection{language=80x86}

\begincode
\begin{CodeListing}[language=80x86, title=Intel 80x86 Assembler Beispiel]
push    ebp       ; save calling function's stack frame (ebp)
mov     ebp, esp  ; make a new stack frame on top of our caller's stack
sub     esp, 4    ; allocate 4 bytes of stack space
\end{CodeListing}
\endcode

\subsection{language=clang}

\begincode
\begin{CodeListing}[language=clang, title=Example C-Code]
#include <stdlio.h>

// This is a hello-world program in C
int main(int argc, char** argv)
{
    printf("Hello, World!\n");
    return 0;
}
\end{CodeListing}
\endcode

\subsection{language=riscv}

\begincode
\begin{CodeListing}[language=riscv, title=Example RISC-V Assembler]
        .data                       ; start of data segment
str0:   .asciz "Hello, World!\n"    ; define string

        .text                       ; start of text segment
        la a0, str0                 ; load start adress of str0 into a0
        li a7, 4                    ; load command 4 into a7
        ecall                       ; environment call
\end{CodeListing}
\endcode

\printindex

\lstlistoflistings
\listoffigures

\end{document}
