\documentclass[
    narrow,
    fontstyle=light,
    babelparam=ngerman
]{elite}

\usepackage[utf8]{inputenc} % Sonderzeichen ermöglichen
\usepackage{blindtext}

% \usepackage[a4paper, margin=1in]{geometry}

\title{S32 \LaTeX{} Document-Class \enquote{elite}}
\author{Christian Schliz und Julian Keck}
\date{\today}

\begin{document}

\maketitle
\tableofcontents
\newpage

\section{Allgemeines}

\section*{Section}
\subsection*{Sub-Section}
\subsubsection*{Sub-Sub-Section}
\paragraph*{Paragraph}

\paragraph*{Textfarben}

\begin{itemize}
    \item \texttt{\textbackslash{}important} \important{Wichtige Hervorhebung}
    \item \texttt{\textbackslash{}highlight} \highlight{Textmarkierung}
    \item \texttt{\textbackslash{}formalWord} \formalWord{Formales Wort}
\end{itemize}

\section{Übersicht über die Textgrößen}

{\tiny tiny}\\
{\scriptsize scriptsize}\\
{\footnotesize footnotesize}\\
{\small small}\\
{\normalsize normalsize}\\
{\large large}\\
{\Large Large}\\
{\LARGE LARGE}\\
{\huge huge}\\
{\Huge HUGE}\\

\section{Elite Definitionen}

\definition{Definition}{%
    Eine Definition wird mit dem Befehl \texttt{\textbackslash{}definition\{\textit{name}\}\{\textit{inhalt}\}}
    erstellt. Diese und weitere Definitionsblöcke basieren auf dem Befehl \texttt{\textbackslash{}abstractDefinition}
    und beinhalten eine Minipage mit dem jeweiligen Inhalt.
}

\defwarn{Warnung}{%
    wird per \texttt{\textbackslash{}defwarn\{\textit{Warnung}\}\{\textit{inhalt}\}} erstellt.
}

\defproblem{Problem}{%
    wird per \texttt{\textbackslash{}defproblem\{\textit{Problem}\}\{\textit{inhalt}\}} erstellt.
}

\defhint{Hinweis}{%
    wird per \texttt{\textbackslash{}defhint\{\textit{Hinweis}\}\{\textit{inhalt}\}} erstellt.
}

\fig{Worsch (\texttt{\textbackslash{}fig\{Unterschrift\}\{Quelle\}})}{./worsch}
\newpage

\section{Wrapfigure Test}

In diesem Abschnitt wird ein normales wrapfig benutzt, Codebeispiel unten auf der Seite.\par

\blindtext
\begin{wrapfigure}{r}{0.25\textwidth}
    \includegraphics[width=0.9\linewidth]{./worsch}
    \caption{GBI Prof.}
    \label{fig:wrapfigure}
\end{wrapfigure}
\blindtext
\blindtext

\begin{CodeListing}[title=manuelle Wrapfig]
\begin{wrapfigure}{r}{0.25\textwidth}
    \includegraphics[width=0.9\linewidth]{./worsch}
    \caption{GBI Prof.}
    \label{fig:wrapfigure}
\end{wrapfigure}
\end{CodeListing}

\newpage

\wrapfig{r}{0.25\textwidth}{Thomas Worsch}{./worsch}

\blindtext

\begin{CodeListing}[title=Macro Wrapfigure]
\wrapfig{r}{0.25\textwidth}{Thomas Worsch}{./worsch}
\end{CodeListing}

\newpage
\section{Algorithmen und so}
\blindtext\blindtext\blindtext\blindtext\blindtext\

\begin{CodeListing}[caption=Beispiel Code-Listing]
\begincode
\begin{CodeListing}[language=java, caption=Beispiel-Code]
public class Example {
    public static void main(String args[]) {
        System.out.println("Hello, World!");
    }
}
\end{CodeListing
\endcode
\end{CodeListing}

\begin{CodeListing}[language=pseudocode]
Function test(a: $\mathbb{N}$) {
    while peter is doof do
        toete peter
        asdfasdf    # yes
        asdfasdf    // test
        asdfasdf
        asdf
}
\end{CodeListing}

\blindtext\n

\begincode
\begin{CodeListing}[language=java, caption=Beispiel-Code]
public class Example {
    public static void main(String args[]) {
        System.out.println("Hello, World!");
    }
}
\end{CodeListing}
\endcode

\blindtext\n

\begincode
\begin{CodeListing}[language=80x86, title=Intel 80x86 Assembler Beispiel]
push    ebp       ; save calling function's stack frame (ebp)
mov     ebp, esp  ; make a new stack frame on top of our caller's stack
sub     esp, 4    ; allocate 4 bytes of stack space
\end{CodeListing}
\endcode

\begincode
\begin{CodeListing}[language=clang, title=Example C-Code]
#include <stdlio.h>

// Das hier ist ein Hello-World Programm in C
int main(int argc, char** argv)
{
    printf("Hello, World!\n");
    return 0;
}
\end{CodeListing}
\endcode

\begincode
\begin{CodeListing}[language=riscv, title=Example RISC-V Assembler]
        .data                       ; start of data segment
str0:   .asciz "Hello, World!\n"    ; define string

        .text                       ; start of text segment
        la a0, str0                 ; load start adress of str0 into a0
        li a7, 4                    ; load command 4 into a7
        ecall                       ; environment call
\end{CodeListing}
\endcode

\printindex
\lstlistoflistings
\listoffigures
\end{document}